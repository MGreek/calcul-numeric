\documentclass{article}
\usepackage{graphicx}
\usepackage{amsmath}
\usepackage{hyperref}

\title{1. Fehleranalyse}

\begin{document}

\maketitle

\section{Gaußsche Eliminationsverfahren}
Der Code für gaußsche Eliminationsverfahren ist \it{\href{https://github.com/MGreek/calcul-numeric/releases/tag/GaussianElimination}{hier}} verfügbar.

\section{Man löse $Ax=b$ und $\tilde{A}x=b$}

$$A=\begin{bmatrix}
10 & 7 & 8 & 7 \\
7 & 5 & 6 & 5 \\
8 & 6 & 10 & 9 \\
7 & 5 & 9 & 10
\end{bmatrix}$$


$$\tilde{A}=\begin{bmatrix}
10 & 7 & 8.1 & 7.2 \\
7.08 & 5.04 & 6 & 5 \\
8 & 6 & 9.98 & 9 \\
6.99 & 4.99 & 9 & 9.98
\end{bmatrix}$$

$$b^T=[32 \ 23 \ 33 \ 31]$$
$$\tilde{b}^T=[32.1 \ 22.9 \ 33.1 \ 30.9]$$

$$s^T=[0.68 \ 1 \ 1.83 \ 0.5]$$

$$\tilde{s}^T=[72.52 \ -117.79 \ 31.59 \ -17.29]$$

Es ist merkwürdig, dass die Matrizen $A$ und $\tilde{A}$ sehr änlich sind aber die Lösungen $s^T$ und $\tilde{s}^T$ ganz verschieden sind.

\end{document}